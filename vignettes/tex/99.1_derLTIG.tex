On the one hand, for any frailty distribution $f ( u_{i} ; \xi )$, 
  the $\alpha$-th moment of $U$ ($\alpha \in \mathbb N$),
  conditional on the data from the $i$-th cluster and on the parameters,
  can be written in the form 
\begin{equation} \label{eq:condEfrailty}
  \E(U^{\alpha} \mid \bm z_i, \bm \tau_i; \bm\psi, \bm\beta, \xi) = \frac{
    \E \left( U^{d_i + \alpha} \exp\left( 
      -U H_{i\cdot,c}(\bm y_i)
%       \sum_{j = 1}^{n_i} H_0(y_{ij}) \exp(\bm x^\top_{ij} \bm\beta) 
      \right)\right)
  }{
    \E \left( U^{d_i} \exp\left(
      -U H_{i\cdot,c}(\bm y_i)
%       \sum_{j = 1}^{n_i} H_0(y_{ij}) \exp(\bm x^\top_{ij} \bm\beta ) 
    \right) \right)}
  \ccom
\end{equation}
with
$H_{i\cdot,c}(\bm y_i) = \sum_{j=1}^{n_i} H_0 ( y_{ij} ) \exp ( \bm x^\top_{ij} \bm\beta )$.

This is a generalisation of a result found by \cite{WangEtal95} which follows from Bayes's formula applied to $f ( u_{i} \mid \bm{z}_{i}, \bm{\tau}_i ; \bm{\psi}, \bm{\beta}, \xi )$ in
\[
 \E \left( U^\alpha \mid \bm z_i, \bm \tau_i ; \bm\psi, \bm\beta, \xi \right) 
  = \int_0^\infty u_i^\alpha f \left( u_i \mid \bm z_i, \bm \tau_i ; \bm\psi, \bm\beta, \xi \right) \mathrm d u_i 
\]
(see Appendix~\ref{app:condEfrailty} for more details).
%   
Now, since the expected values in the right-hand side of Equation~\ref{eq:condEfrailty}
  can be written in terms of derivatives of the Laplace transform
\[
  \E \left( U^q \exp(-sU) \right) = (-1)^q \mathcal L^{(q)}(s), \qquad q, s \geq 0,
\]
we have that
\begin{equation} \label{eq:dLTrecurs}
  \mathcal{L}^{( d_{i} + \alpha )} \left( 
    H_{i\cdot,c}(\bm y_i)
%     \sum_{j = 1}^{n_{i}} H_{0} ( y_{ij} ) \exp ( \bm{x}^\top_{ij} \bm{\beta} ) 
    \right) =
    ( - 1 )^{\alpha} \E ( U^{\alpha} \mid \bm{z}_{i}, \bm{\tau}_i; \bm{\psi}, \bm{\beta}, \xi ) 
    \ \mathcal{L}^{( d_{i} )} \left( 
    H_{i\cdot,c}(\bm y_i)
%     \sum_{j = 1}^{n_{i}} H_{0} ( y_{ij} ) \exp ( \bm{x}^\top_{ij} \bm{\beta} ) 
    \right).
\end{equation}

On the other hand, if $U \sim \textrm{IG}^\star (\theta)$, 
  then it is easy to show (Appendix~\ref{app:condIG})
  that the conditional distribution of $U$ given the data and the parameters is a 
  generalised inverse Gaussian distribution:
  \[
   U \mid \bm{z}_{i}, \bm \tau_i; \bm{\psi}, \bm{\beta}, \theta
    \sim \mathrm{GIG} (\gamma_{_{\mathrm{GIG}}}, \delta_{_{\mathrm{GIG}}}, \theta_{_{\mathrm{GIG}}} )
  \]
  with
  \begin{align} \label{eq:GIGpar1}
    \gamma_{_{\textrm{GIG}}} &= d_i - \frac12\ccom \\
    \theta_{_{\textrm{GIG}}} &= \frac1{2 \theta} + H_{i\cdot,c}(\bm y_i), \\
%       \sum_{j = 1}^{n_{i}} H_{0} ( y_{ij} ) \exp ( \bm{x}^\top_{ij} \bm{\beta} ), \\
    \delta_{_{\textrm{GIG}}} &= \frac1{\sqrt{2 \theta}}\cdot
    \label{eq:GIGpar3}
  \end{align}

Hence \cite[Section A.3.6]{Hougaard00}
\begin{equation} \label{eq:condEfIG}
  \E ( U^{\alpha} \mid \bm{z}_{i}, \bm{\tau}_i; \bm{\psi}, \bm{\beta}, \xi ) = 
    \left( \frac{\theta^{1 \slash 2}_{_{\textrm{GIG}}}}{\delta_{_{\textrm{GIG}}}} \right)^{- \alpha}
    \frac{K_{\gamma_{_{\textrm{GIG}}} + \alpha} ( 2 \delta_{_{\textrm{GIG}}} \theta^{1 \slash 2 }_{_{\textrm{GIG}}} )}
          {K_{\gamma_{_{\textrm{GIG}}}} ( 2 \delta_{_{\textrm{GIG}}} \theta^{1 \slash 2 }_{_{\textrm{GIG}}} )}
    \cdot
\end{equation}

Combining (\ref{eq:dLTrecurs}) and (\ref{eq:condEfIG}), Equation~\ref{eq:invGauss} is deduced.
\qed
